\iflanguage{ngerman}
{\chapter{Verwandte Arbeiten}}
{\chapter{Related Work}}
\label{sec:related}

Sowohl das Erzeugen von Vector-Graphiken auf Fragmentbasis als auch die Darstellung von Daten in Graphen (Plots) finden in der Wissenschaft Anwendung und werden auch in vielen Arbeiten erforscht. Die Kombination hingegen findet bisher keine große, wissenschaftliche Aufmerksamkeit. Das kann möglicherweise daran liegen, dass in allen Anwendungsfällen mit statischen Daten geometriebasierte Plots sinnvoller wären, da die Geometrien nicht in jedem Frame neu berechnet werden müssen. Somit habe Ich keine direkt Verwandten Arbeiten gefunden, die dasselbe Ziel verfolgen würden, sondern Arbeiten die in Teilen Verwandt sind.

\section{Vektor-Graphik}
In der Arbeit [] wird die Darstellung von Vektor-Graphiken beschrieben. Beschriebene Techniken beinhalten Prefiltering, SDFs und

