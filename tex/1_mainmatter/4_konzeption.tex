\iflanguage{ngerman}
{\chapter{Konzeption}}
{\chapter{Concept}}
\label{sec:concept}

\section{Referenzsystem}
Alle Elemente des Plots, die auf der zugrunde liegenden Geometrie dargestellt werden brauchen zur Bestimmung ihrer Position ein Referenzsystem.
Dieses Referenzsystem ist im Texturraum der Geometrie.
Also können wir z.B. die UV-Koordinaten nutzen, die uns GLSL bereitstellt.
Der UV-Koordinatenraum erstreckt sich von (0.0, 0.0) zu (1.0, 1.0).
Ich bevorzuge jedoch ein Referenzsystem, das sich von (-1.0, -1.0) bis (1.0, 1.0) erstreckt und bei dem der Punkt (0.0, 0.0) genau in der Mitte liegt.
Das ist auch das Referenzsystem von dem Ich in der restlichen Arbeit ausgehen werde.

\section{Graphen}
Es soll die Möglichkeit geben, statt nur einzelner Punkte, auch Kontinuierliche Graphenlinien zu Zeichnen.
Der einfachste Weg so eine Linie darzustellen ist, das einfache erzeugen eines Liniensegments zwischen zwei, nebeneinander liegenden Punkten im Datensatz für den Graphen.

\section{Interaktion}
Die Implementierte Demo soll dem Nutzer die Möglichkeit geben beliebig viele Graphen im selben Plot darzustellen und diese auch individuell anpassen zu können.
Der Nutzer muss auch in der Lagen sein im 3D-Raum mit dem Plot zu interagieren durch einen adaptive Grid-Auflösung, die sich an der Distanz zur Kamera orientiert.

\section{Dekoration}
