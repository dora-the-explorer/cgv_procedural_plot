\iflanguage{ngerman}
{\chapter{Einleitung}}
{\chapter{Introduction}}

\label{sec:introduction}

Traditionell werden Plots in der Computergraphik aus Einzelteilen zusammen gebaut.
Dabei hat jedes Element im Plot eine eigene Geometrie, eine Achse ist zum Beispiel ein sehr langes Rechteck.
Diese Darstellung hat einige schwierige Probleme, wie z.B. Z-Fighting.
Dabei entstehen visuelle Artefakte bei Elementen die sich direkt Überlappen.
Ein anderes Problem ist, dass bei der Darstellung von Echtzeitdaten, sämtliche Geometrien neu berechnet werden müssen.
Weniger ein Problem als eine neutrale Tatsache ist, dass diese Traditionellen Plot-Renderer die CPU stärker beanspruchen, um die Geometrieformen zu berechnen.
\par
Die in dieser Arbeit betrachtete Alternative, ist das vollständige Generieren von Plots im Fragment-Shader.
Dabei werden alle Elemente des Plots im Texturraum und auf Fragmentbasis erzeugt.
Z-Fighting ist damit komplett irrelevant und die Performance bleibt beim nachladen von neuen Elementen gleich.
Zudem verlagert dieser Ansatz fast alle Arbeit von CPU auf die GPU, was für manche Anwendungsfälle von Nutzen sein kann.


