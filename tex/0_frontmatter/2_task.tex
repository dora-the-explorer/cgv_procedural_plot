
\course{Informatik}
\matriculationyear{2018}
\issuedate{25.04.2022}
\duedate{11.07.2022}
\professor{Prof. Dr. Stefan Gumhold}


\begin{task}[] % If needed, a custom title can be provided between the brackets
	\minisec{Motivation}\smallskip
	Immersive Visualisierungen ermöglichen intuitives Erkunden und verstehen komplexer Daten,
	jedoch müssen häufig auch klassische Ansichten wie Zeitreihenplots zu Navigations- und Übersichtszwecken
	eingebettet werden. Die Teilelemente dieser Plots werden traditionell mittels eigener Geometrie dargestellt.
	Mit dieser Technik muss bei jeder Veränderung des Plots mindestens die Geometrie der betroffenen
	Teilelemente neu erzeugt werden. Bei Performanz-kritischen Echtzeitanwendungen, wie dem Streamen von
	Live-Daten oder der dynamischen Interaktion mit dem Plot, will man jedoch diese Neuerzeugungen vermeiden.
	Ein weiteres Problem sind visuelle Artefakte, die durch z-Fighting beim Layering der Geometrien entstehen können.
	Eine mögliche Alternative ist das prozedurale Erzeugen der Plotstrukturen im Texturraum,
	was sich auf modernen GPUs effizient auf Fragmentbasis implementieren lässt. Durch das Erzeugen im
	Texturraum kann die dynamische Anpassung von Skalierung und Granularität der Achsen zur Interaktion mit
	dem Plot, sowie das Nachladen von Live-Daten, effizient implementiert werden. Bei diesem Ansatz werden
	alle Plotstrukturen auf einer einzigen Geometrie dargestellt, wodurch z-Fighting vermieden wird.
	Ein weiterer Vorteil ist die triviale Umsetzung von Antialiasing beim Arbeiten auf Fragmentbasis.
	
	\minisec{Ziel der Arbeit}\smallskip	
	Ziel der Bachelorarbeit ist es, eine Bibliothek zur Texturraum basierten Darstellung von Plots mit
	OpenGL zu implementieren und abschließend die Implementierung in Bezug auf Funktion und Performance
	zu analysieren. Dazu wird auch nach Strategien zur Dekoration von Plots in existierenden Plotbibliotheken
	gesucht und die Umsetzbarkeit dieser im Texturraum diskutiert. Die Arbeit beschränkt sich dabei auf 2D Plots.
	
	\newpage	
	
	\minisec{\focusname}\smallskip
	\begin{itemize}
		\item{Literaturrecherche:}
		\begin{itemize}
			\item{Prozedurales Rendering mit Fokus auf Texturraum}
			\item{Strategien und Algorithmen zur Dekoration von Plots in etablierten Tools wie GnuPlut
			und Excel oder Algebrasystemen wie Matlab, Mathematica und Maple}
		\end{itemize}
		\item{Analyse und Diskussion zur Anwendbarkeit existierender Methoden im Fragment Shader}
		\item{Auswahl (ggf. Neudesign) und Implementierung geeigneter Methoden in Form eines Renderers
		für das CGV-Framework mit folgenden Eigenschaften:}
		\begin{itemize}
			\item{Konfiguration über Render Styles}
			\item{Layer-Prinzip zur Organisation der Plotkomponenten}
			\item{Punkt- und linienbasierte Dekorationen}
			\item{Schriftunterstützung für Achsen- und Tickmark-Labels}
		\end{itemize}
		\item{Evaluation und Diskussion des entwickelten Ansatzes}
		\begin{itemize}
			\item{qualitative Analyse der Renderingergebnisse (Artefakte etc.)}
			\item{quantitative Analyse der Renderperformanz der Implementierung}
		\end{itemize}
	\end{itemize}
	
	\minisec{Optionales}\smallskip
	\begin{itemize}
		\item{blickpunktabhängige Adaption von Tickmarks und Gittern}
		\item{blickpunktabhängige Textausrichtung}
		\item{Titel, Untertitel oder komplett frei platzierbare Label mit beliebigem Text}
	\end{itemize}
	\bigskip
\end{task}

